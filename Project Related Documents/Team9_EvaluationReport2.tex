\documentclass[11pt]{article}

\begin{document}
\title{Evaluation report 2\\ Hearing Colours In Space}
\author{Harpo 't Hart \\ Phillipe Louchtch}
\date{06-01-2013}
\maketitle

\tableofcontents

\section{introduction}
\subsection{Purpose of evaluation}
This was an evaluation of the full prototype of the device. The purpose was to see if users were able to intuitively use the functions of the GUI and the voice commands. As well as to see if the changes made since the last test did actually improve the interaction with the device.
\subsection{Goals of evaluation}
The goals of the evaluation were to test the visibility of the functions in the GUI. As well as the intuitiveness of the voice commands. And an overall user satisfaction using the device. This user evaluation focussed on the following questions:
\begin{itemize}
\item Were the test persons able to do the tasks specified in 2.4 successfully on the GUI and voice commands?
\item Did the changes made in the sound synthesis provide better understanding or a more pleasant listening experience for the users?
\item Did the test persons feel they had enough influence on the device?
\end{itemize}
\subsection{Test subject profiles}
For this session three test persons were present Absaline, Joav and Rik. These test subjects had no prior experience in testing software.
\begin{center}
    \begin{tabular}{| l | l | l | l|}
    \hline
     	Test person & Absaline & Joav & Rik \\ \hline
	Age & 25 & 26 & 25 \\ \hline
	Gender & Female & Male & Male \\ \hline
	Background & Philosophy student & Sound designer & Art student \\ \hline
	Computer skills & intermediate & expert & intermediate \\ \hline
    \end{tabular}
\end{center}
\subsection{Prototype}
For this test a full prototype of the device was used. All the functionality was operational. Extra attention was given to the graphical user interface and the voice commands. These were all implemented as they should in the final product. There was one element missing in the prototype that was mentioned in the paper design. That is the option to detect only distances and no colours. 
An important new feature to this prototype was the introduction of a new sound and mapping technique. This was a granular synthesizer that produces clouds of sound in which the pitch and the spatial position of the grains correspond with a specific colour in a target box. The difference in mapping colour to sound was that colours that are close to each other generate tones with an harmonic relation instead of a chromatic relation.
\subsection{Test environment}
For a test environment we chose the public library in Amsterdam (OBA). The lighting in the library is very equal and quite bright. On the walls are coloured paintings and pictures. The walls themselves are white. From the first user evaluation it became clear that lighting was an important issue. For this device to detect colours properly it was needed that the light is as equal and bright was possible.

\section{Method}
\subsection{Usability testing method}
The method used for this test was cooperative evaluation. Each participant evaluated the interface in one session. The three subjects did the test independently of each other. The subjects didn't give suggestions or helped each other during the testing. The test was finished when the test subjects finished their tasks or didn't know how to continue.
\subsection{Role of observer}
two observers were present when the subjects did the evaluations. The observers explained what was expected of the participants before they took the test. During the test the observers asked of the test persons to keep talking and thinking aloud. 
\subsection{Description of test procedure}
At first the observers explained the subjects the procedure of the testing, a short explanation of the device. Then the observers told the test subjects about the think aloud method and that they would be transcribing the comments the test persons would make. The test persons were one by one introduced to the tasks just before they took the test.  The first test person started the tasks while the observers were observing the test persons and making notes of their actions and comments. After the test the observers had a small discussion with the test person about what they liked or not about the interface and what in their opinion could be improved. This was done in the same way with the two other test persons.
\subsection{Tasks}
The test subjects were asked to perform the following tasks: 
\begin{enumerate}
\item Change amount of targets .
\item Enabling "zoom".
\item Changing the volume.
\item Changing the sound.
\end{enumerate}
These tasks were to be carried out twice in two different ways: Once through the GUI and also by using voice commands. In the test process all the tasks were first done on the GUI and after that by using voice commands. Extra attention was given to the 'change the sound' task. When the test persons did this the observers asked of them what the test persons thought of the different sounds. Furthermore they asked if one gave a better feeling of mapping colour than the other and if they found one more pleasing than the other. 

\section{Results}
\subsection{Summary of test subject 1 (Absaline)}
Absaline is not an expert computer user, but she uses computers on a regular basis. When using the graphical interface she performed the tasks quite quickly without much errors. There was one feature she didn't get so quickly: the zoom option. She could find the button and press it, but she wasn't aware of the effect in had. So although she was very well able to perform the action on the interface she didn't understand what that functionality was for. The voice commands didn't go as smooth. The first action she tried, to alter the volume, worked right away. After that she had to speak the commands quite often to get the right result. She could find the list with voice commands easily, but the software didn't react to her voice so well.
\newline
In the discussion after the test she told that she didn't quite understood what she was hearing. She clearly got the notion that she had an influence on the sound, but she couldn't tell what it was. What she did understand was the spatialization of sound. When she saw a coloured object going from left to right she also clearly heard that in the spatialization of the sound. From the two types of sound she liked the new granular sound better than the additive sound, because it was more pleasant to listen to.
\subsection{Summary of test subject 2 (Joav)}
Joav is an experienced computer user. He could find all the functions in the GUI quickly. The voice commands didn't work as well. He could find the list of voice commands easily and speak them out loud, but the system didn't react to his voice so well. He had to repeat voice commands quite often to get the right result. 
\newline
In the discussion after the test he told that it was quite clear to him that the device reacts to colour.  He also was very aware of the spatialization of the sound. He said that he would like to have some reverb on the sounds to enhance the spatial effect. He also said that he would like to see in the interface what the kinect records, so you can know what the kinect is doing. When asked what he thought about the granular sound versus the additive sound he told that he found the granular sound more pleasing to the ear, but a lot less clear than the formant sound.
\subsection{Summary of test subject 3 (Rik)}
Rik is not an expert computer user. When using the graphical interface most of the tasks were easy for him to accomplish. There was one option that distracted him. In the interface there is an option to choose between two colour averaging algorithms. He didn't have to use it, but it disturbed him that when choosing the option he didn't perceive any effect of his action. With the voice commands he had much success, but in quite a creative way. When he found that his voice commands didn't have the right effect he held his hands in such a way (like a muslim praying) that the sound was reflected to the microphone on the kinect. After he found this out almost all the voice commands worked in one try.
\newline
In the discussion after the test he told he felt there was a relation of the movements of his head with the sound he was hearing. It isn't immediately clear that the device reacts to colours, but it's inviting to try and find out what it does. When asked about the two types of sound he said that although the granular sound is much more pleasant, the additive sound is a lot clearer to detect what is going on.

\section{Discussion}
\subsection{Evaluation of results}
The users were all three very well able to work with the GUI. There were some options in the menu that didn't feel logical to the test persons. The affordance of these options was unclear. The test persons could find the options and perform the actions, but didn't get the notion of the things that had changed because of their actions. 
\newline
The voice commands were quite a big obstacle for the test persons, although most of them found the option real fancy. Because the bright finding of test person three (Rik) we discovered that this has to do with the placing of microphones on the kinect. Because the microphones are on top of the head of the user they do not pick up much signal of his/her voice.
\newline
On a general level the users did understand that the sounds they were hearing had to do with what they saw. But the linking of the sound to colour was less intuitive. This may have to do with the mapping of the colour to sound. What was quite clear was that the mapping of the sonochromatic scale in combination with the additive sound was a clearer mapping than the Skrjabin colour scale to the granular sound. 

\subsection{Recommended adjustments} 
The things that can be improved:
\begin{itemize}
\item Position the microphones for the voice recognition better to pick up the voices of the user.
\item Find mappings that are clear and pleasing at the same time.
\item Make visible what the kinect registers in the visual interface.
\end{itemize}



\end{document}