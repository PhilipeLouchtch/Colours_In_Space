\documentclass[11pt]{article}

\begin{document}
\title{Evaluation report 1\\ Hearing Colours In Space}
\author{Harpo 't Hart \\ Phillipe Louchtch}
\date{24-11-2012}
\maketitle

\tableofcontents

\section{introduction}
\subsection{Purpose of evaluation}
This was the first evaluation of the first prototype of the device. In this early state of the design process the purpose of this evaluation was to see if the test subjects felt invited to play around with the device, if they intuitively understood how the device worked and to ask the test persons for feedback. The goal was to get an overall impression to see if the design was going the right direction so far.
\subsection{Goals of evaluation}
The goal of this evaluation was to test the Learnability, Intuitiveness and user satisfaction for the device. The test carried out focussed on the following question:
\begin{itemize}
	\item Was the device inviting for people to play around with and discover its possibilities for themselves?
	\item Did the users feel that the sound and the colours were related and if so how did they discover or experience this relationship?
	\item Did the users enjoy making sound in this way and did they feel they had enough influence on the device?
\end{itemize}
\subsection{Test subject profiles}
The test subjects were 2 young men (Marco and Wieger) with an artistic background that knew each other very well. Both subjects had no prior experience in testing software.
\begin{center}
    \begin{tabular}{| l | l | l |}
    \hline
     	Test person & Marco & Wieger \\ \hline
	Age & 26 & 25 \\ \hline
	Gender & Male & Male \\ \hline
	Background & Musician & Theatrical performer \\ \hline
	Experience with kinect & novice & novice \\ \hline
    \end{tabular}
\end{center}
\subsection{Prototype}
The prototype used was a horizontal prototype of the system. The features that were essential for the device to work were present. The prototype was meant to see if the users got the general concept of the device.
The functionalities that did work were:
\begin{itemize}
\item The mapping of colour to sound
\item The possibility to change the resolution of the colours through a graphical user interface
\item 2 types of sounds to choose from
\end{itemize}
The mapping of colour to sound was done in a basic way. There were twelve colours mapped to twelve tones in an octave. The sound was deliberately kept simple to make it clear to see if the pitch to colour mapping was perceived by the test persons. The two sounds the test persons could choose from was a simple sine wave or a simple additive sound that got a brighter timbre as the colour got brighter.
\subsection{Test environment}
The test was carried out in a recording studio. It was quite a small room, lit by halogen lamps. It was quite dark so we had to use extra lights to reach the desired level of lightness. The walls were of all the same colour and on the wall there were some coloured paintings and posters.

\section{Method}
\subsection{Usability testing method}
The subjects tested the device using the co-discovery method. The two subjects took the test together at the same time. There was only one device available so the test persons took turns using it. They stayed in the room when the other test person used the device allowing them to discuss their findings and investigate the device and its possibilities together.
\subsection{Role of observer}
The observers were making notes of the remarks the test persons made while trying the device. The observers were also making an effort to keep the test persons talking while they were taking the test. Also did they make suggestions and ask questions.
\subsection{Description of test procedure}
The test started with a short explanation of the device. Then the observers told the test subjects about the think aloud method and that they would be transcribing the comments the test persons would make. After that the first test subjects were given the device to try out. The second test subject was in the room with him to see what happened. Then the first subject was free to explore the device. After he was done the second test person got the device while the first test person was present. 
\subsection{Tasks and suggestions}
The test subjects were asked to perform the following tasks:
\begin{enumerate}
\item Walk around the room and listen to the sounds
\item Find a spot to look at that sounds the most pleasing
\item Change the amount of mappings/targets/boxes by using the GUI
\item Change the sound
\end{enumerate}
The last task, to change the sound, wasn't so much about the action of changing the sound through the interface but to check if the test subjects preferred one type of mapping over the other. 
Next to these tasks the observers made suggestions to the test persons to, for instance, take a little more distance from some object or find a space with more light. This was meant to make the test persons look at different things and experience the whole range of possibilities of the device.
\section{Results}
\subsection{Transcription of remarks per test subject}
\subsubsection{Task 1}
\begin{center}
    \begin{tabular}{| p{6cm} | p{6cm} |}
    \hline
  	Marco & Wieger \\ \hline
	'If you light the surroundings equally it's clear to hear what happens' & 'When I move my 			head slower i have the feeling that i have more influence on the sound' \\
	'A lot of sound for your ear' & 'I didn't get the spatial dimension of the sound, although I did 		have the feeling that I was listening to multiple colours at once'\\
	'It's kind of trippy' & \\ \hline
	
    \end{tabular}
\end{center}
\subsubsection{Task 2}
\begin{center}
    \begin{tabular}{| p{6cm} | p{6cm} |}
    \hline
  	Marco & Wieger \\ \hline
	'Cool that you can influence the sound by moving around' & 'I would have liked to have a 			room with more distinct colours' \\
	'When you look at a spot in mainly the same colour you get a quite dissonant sound' & 'You 		hear difference between surfaces with much colours and surfaces that are monochrome' \\ 			\hline
    \end{tabular}
\end{center}
\subsubsection{Task 3}
\begin{center}
    \begin{tabular}{| p{6cm} | p{6cm} |}
    \hline
  	Marco & Wieger \\ \hline
	'Five colours sounds more chaotic than three' & 'Not so much chaotic, you only hear more 			sounds'\\
	'Three colours are more intuitive to use'  & ' You should be able to make more steps 				instead of only choosing between five and three' \\
	'I have the feeling that I have more influence on the sound with three colours' & 'I have the 			feeling that I have just as much influence with three as with five colours' \\
	'Visual interface looks intuitive because of the image' & 'I would like to see numbers at the 			scrollbar to know what I'm doing' \\ \hline
    \end{tabular}
\end{center}
\subsubsection{Task 4}
\begin{center}
    \begin{tabular}{| p{6cm} | p{6cm} |}
    \hline
  	Marco & Wieger \\ \hline
	'I do not hear so much difference' & 'When you look at a brighter surface it actually sounds a                bit brighter' \\ 	
	'It would be nice if I could choose between more different sounds' &  \\\hline  		
    \end{tabular}
\end{center}
\subsubsection{Miscellaneous}
\begin{center}
    \begin{tabular}{| p{6cm} | p{6cm} |}
    \hline
  	Marco & Wieger \\ \hline
	  & 'This would be nice to do a performance with if it is further developed' \\ \hline
    \end{tabular}
\end{center}	

\section{Discussion}
\subsection{Evaluation of results}
This test shows that on the whole the device is inviting and fun to work with and explore its possibilities and the users felt that the sounds they were hearing had to do with what they saw. The test persons had fun using the device and showed that they had a global understanding of what was happening. On the more detailed level there were a lot of things that didn't fit with the expectations or requirements of the users. 
The most named thing was the feeling of influence. The test persons wanted to have more control on the colours they were hearing and wanted to have more choice of sounds. 
 Another problem was that if the lighting wasn't that good the colours sounded less differentiated than when the lighting was better. 
 A third  problem was that when the test persons looked at an object with for instance multiple shades of blue the sound got really dissonant. This didn't feel intuitive, since the surface they looked at looked quite harmonious, but the sound was disharmonious.

\subsection{Recommended adjustments} 
The things that will be improved in the next prototype:
\begin{itemize}
\item Another mapping of colour to sound will be sought to make more harmonious colours sound more harmonious.
\item In the GUI the scrollbar to select the number of boxes will have numbers to indicate how many boxes are selected.
\item A lamp will be mounted on the device to make sure the lighting is equal everywhere you look.
\item Different sounds will be made available for the users to choose from.
\end{itemize}



\end{document}